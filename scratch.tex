At the time, the R-tree  was a great solution to the problem of efficiently searching
for spatial data in a database; however, there were certain issues with the original
implementation. For instance, the complexity of the insertion algorithm
depends on the complexity of the node splitting function, which is typically
quadratic. Similarly, the heuristics for node splitting affects the efficiency of 
the search algorithm. These challenges are discussed in detail in 
Section~\ref{sec:impchal}. In the below sections, we discuss the main R-Tree
variants that seek to solve some of the disadvantages of the R-tree.

\subsubsection{$R^{+}$-Trees}
Shortly after Guttman's paper in 1984, Sellis, Roussopoulos, and Faloutsos proposed 
the $R^{+}$-tree. The problem they sought to solve was the performance degredation 
during a range search due to high MBR overlap. Basically, the $R^{+}$-tree tries to 
reduce the number of paths explored (and consequently the number of I/O operations) 
during a search by disallowing overlap in MBRs in internal nodes of the same level. 
This means that objects that span across multiple MBRs are stored in multiple leaves. 
This duplication has several consequences that affect searches, deletion, and insertion.

% Search & Delete
Searches in $R^{+}$-trees are performed the same as in the R-tree case, except
redundant entries must be eliminated during a range query. However, in point queries
$R^{+}$-trees are guaranteed to only traverse one path since no MBRs overlap in 
internal nodes. Conversely, deletion is slightly more complex than in the R-tree case
since there may be more than one entry that must be deleted. This means that multiple 
leaves may be visited. Essentially, the $Delete$ algorithm traverses the tree using
the MBR of the entry to delete as the search parameter and when it reaches a leaf, it 
removes the corresponding entry and propagates the change upward. In 
\cite{sellisroussopoulosfaloutsos87} the authors do not discuss how they handle underfull nodes.


% Insertion
\begin{figure}[t!]
\begin{algorithmic}
	\Function{Insert}{$R$, $E$}
		\If{$R$ is not a leaf}
			\ForAll{Entries $X$ in $R$}
				\If{$X.mbr$ overlaps $E.mbr$}
					\State \Call{Insert}{$X.p$, $E$}
				\EndIf
			\EndFor
		\Else
			\If {$R$ is full}
				\State \Call{SplitNode}{$R$}
			\Else
			\EndIf
		\EndIf
	\EndFunction
\end{algorithmic}
\caption{Pseudocode for inserting an entry E given a $R^{+}$-tree rooted at R}
\label{fig:R+_Tree_Insert}
\end{figure}

Insertion is slightly different than the R-tree. Since MBRs do not overlap, an
object may be added to more than one leaf node, which is not the case in an R-tree. 
Thus, the insertion algorithm in Figure~\ref{fig:R+_Tree_Insert} is used. We see that
the function recursively inserts $E$ into all leaves with overlapping MBRs instead of 
using path traversal to select just one. Note that the algorithm $SplitNode$ is 
also different from the R-tree algorithm due to the disjoint MBR requirement because
changes may require downward in addition to upward propagation. 

More specifically, the $SplitNode$ algorithm takes care to partition the entries in 
the node into two new nodes with non-overlapping MBRs and propagates this change 
downward and upward, if necessary, using recursive calls to $SplitNode$ on the node's
children and parents, respectively. For brevity, we do not include the $SplitNode$ 
algorithm, but we discuss the $Partition$ algorithm of \ref{fig:R+_Tree_Partition} 
used to determine the regions described by the new nodes in more detail. Basically, 
costs for cuts in each dimension are calculated using $Sweep$. $Sweep$ calculates the
cost based on resultant dead space and number of rectangle splits. The cut that has 
the smallest cost is chosen and the entries in the node to be split are placed 
according to their MBRs. 


\begin{figure}[t!]
\begin{algorithmic}
	\Function{Partition}{$S$, $ff$}
		\If{No partition required}
			\State {$R \Leftarrow$ Node containing $S$}
			\Return {($R$, empty)}
		\EndIf
		\State $O_{x} \Leftarrow$ minimum x coordinate of all r in $S$
		\State $O_{y} \Leftarrow$ minimum y coordinate of all r in $S$
		\State $(C_{x}, x_{cut}) \Leftarrow$ \Call {Sweep}{"x", $O_{x}$, $ff$}
		\State $(C_{y}, y_{cut}) \Leftarrow$ \Call {Sweep}{"y", $O_{y}$, $ff$}
		\Comment{Starting from $O_{x}$ or $O_{y}$ on its respective axis, pick 
		the first $ff$ rectangles sorted on the input axis. Return the 
		cost to split along this axis and the maximum value of the cut}
		\State $cut \Leftarrow min(Cx, Cy)$
		\State {Distribute $S$ according to cut}
		\State $R \Leftarrow$ Node containing entries in first subregion of cut
		\State $S' \Leftarrow$ Set of rectangles not in $R$ 
		\Return ($R$, $S'$)
	\EndFunction
\end{algorithmic}
\caption{Pseudocode for partitioning a set of rectangles $S$ and fill factor ff into
	a node $N$ containing the rectangles of the first subregion and set $S'$ of
	the remaining rectangles for the $R^{+}$-tree}
\label{fig:R+_Tree_Partition}
\end{figure}

% Insertion
Insertion is slightly different than the R-tree. Since MBRs do not overlap, an
object may be added to more than one leaf node, which is not the case in an R-tree. 
Thus, the insertion algorithm in Figure~\ref{fig:R+_Tree_Insert} is used. We see that
the function recursively inserts $E$ into all leaves with overlapping MBRs instead of 
using path traversal to select just one. Note that the algorithm $SplitNode$ is 
also different from the R-tree algorithm due to the disjoint MBR requirement because
changes may require downward in addition to upward propagation. This means that insertion
could be potentially very costly.

More specifically, the $SplitNode$ algorithm takes care to partition the entries in 
the node into two new nodes with non-overlapping MBRs and propagates this change 
downward and upward, if necessary, using recursive calls to $SplitNode$ on the node's
children and parents, respectively. For brevity, we do not include the $SplitNode$ 
algorithm, but we discuss the $Partition$ algorithm of \ref{fig:R+_Tree_Partition} 
used to determine the regions described by the new nodes in more detail. Basically, 
costs for cuts in each dimension are calculated using $Sweep$. $Sweep$ calculates the
cost based on resultant dead space and number of rectangle splits. The cut that has 
the smallest cost is chosen and the entries in the node to be split are placed 
according to their MBRs. 

% Performance comparison
Compared to R-trees, $R^{+}$-trees have better search performance in certain cases,
such as when there are many small objects and a few large objects or during point 
queries, according to the analysis performed in \cite{sellisroussopoulosfaloutsos87}. 
However, there are also cases in which $R^{+}$-trees have worse search performance. 
For instance, when there are many large objects in the database that span across many
MBRs, one object may be replicated across many different nodes which causes range
queries to be less efficient.





