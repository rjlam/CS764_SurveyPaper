As we have toured the classic enhancements and improvements to R-tree performance, we have encountered experimental data hinting towards when and how R-trees perform well and why.
In this section we discuss the papers that took that information and worked towards a theory of R-trees.
There are four parts to our discussion.
First we discuss the \emph{cost model}---or in other words, what machine-level operations actually take the most time in an R-tree?
The obvious answer is I/O costs, but spatial queries can have intense CPU requirements.
Second we consider the data model: what are natural distributions for R-tree data sets?
What data seems particularly ``bad'' or ``good'', if any.
From this foundation we move to our third part, in which we introduced the optimal R-tree \cite{argeberghaverkortyi04}, a proven-optimal R-tree.

\subsection{Cost model}
Spatial data lacks a strict ordering---this seems to be the main reason why it is so ``complicated'' to work with \cite{stuff}.
The R-tree's many advancements have allowed it to perform well in practice \cite{thebook}, but there is an underlying question: how do we know if the R-tree is organized well?
We have seen that imposing a linear ordering can greatly improve performance \cite{kamelfaloutsos94} and there are still new results and questions about possible orderings \cite{haverkortwalderveen11}.
Indeed, the initial models for R-trees were largely dependent on the R-tree's construction; but this quality is incidental to the data \cite{see:list:in:theodoridisstefanakissellis}.

We move past ``index-dependent'' models, for Theodoridis et. al.\ \cite{theodoridissellis96,theodoridisstefanakissellis00} developed a cost model for \emph{range} and \emph{join} queries that depended only on some parameters of the data.
In particular, it depended on the number of elements in the dataset (cardinality) and the spatial density of those rectangles.
Moreover, this was one of the first papers to move beyond the assumption that the data was uniformly distributed.
A key concept was the notion of a \emph{density surface}, a partition of the data space into smaller chunks in which density is estimated---a density histogram, really.
Other authors focused on fractal dimensions [TO ADD].

\subsubsection{Spatial Histograms}
In practice, of course, one needs to maintain useful meta-data for query optimization \cite{chaudhuri98}, and the histogram is one of the main tools for that \cite{poosalahaasioannidisshekita96}.
R-trees can be used \emph{as} spatial histograms \cite{achakeevseeger12}, and serves to estimate a cost of a \emph{range} query.
Theis spatial histogram is to partition the data rectangles $r_1,\ldots,r_N$ into buckets, each bucket maintaining the average $x,y$ lengths of its constituents $r_i,\ldots,r_j$ and the bucket's density.
Implied by its having a density, the bucket induces an MBR over its rectangles $r_i,\ldots,r_j$.
In in their paper \cite{achakeevseeger12}, the authors decide on an equi-depth histogram.
This is complicated by the fact that computing their optimal partition is $\NP$-hard \cite{muthukrishnanpoosalasuel99}.
There design, guided by the cost-model of \cite{theodoridissellis96}, was thus computed with heuristics; the details are in \cite{achakeevseeger12a}.

There also exist histograms for the purpose of estimating, say, a join \cite{aboulnaganaughton00}.
Also takes into account other stuff.
Some experimental models: \cite{aboulnaganaughton00}, \cite{anyangsivasubramaniam01}, \cite{achakeevseeger12,achakeevseeger12a} has good datasets discussion, see for ``data model''.

\subsection{Data model}
Are there data models that are good or bad? Stabbing number (computing it is hard, find that math paper), assuming squareness (an early assumption), other things?
What sort of distributions are typically considered?
Compare/contrast with B-tree distributions.
Theory of hilbertization \cite{haverkortwalderveen11}.

\subsection{Optimal R-tree}
Optimal R-tree \cite{argeberghaverkortyi04}.
Optimal R-tree, high-level \cite{yi12}.
Optimizing for minimal stabbing number \cite{bergkhosraviverdonschotweele11}.
