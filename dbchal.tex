As we have toured the classic enhancements and improvements to R-tree performance, we have encountered experimental data hinting towards when and how R-trees perform well and why.
In this section we discuss the papers that took that information and worked towards a theory of R-trees.
There are four parts to our discussion.
First we discuss the \emph{cost model}---or in other words, what machine-level operations actually take the most time in an R-tree?
The obvious answer is I/O costs, but spatial queries can have intense CPU requirements.
Second we consider the data model: what are natural distributions for R-tree data sets?
What data seems particularly ``bad'' or ``good'', if any.
From this foundation we move to our third part, in which we introduced the optimal R-tree \cite{argeberghaverkortyi04}, a proven-optimal R-tree.
Lastly, we consider analytic methods determined to compute the cost of particular queries.

\subsection{Cost model}

\subsection{Data model}
A model for the prediction of R-tree performance \cite{theodoridissellis96}, and a later paper \cite{theodoridisstefanakissellis00}.

\subsection{Optimal R-tree}
Optimal R-tree \cite{argeberghaverkortyi04}.
Optimal R-tree, high-level \cite{yi12}.


\subsection{Query analytics}
