% Overview 
To solve the problem of performing efficient searchs on spatial data, 
Guttman proposed the R-tree, which inspired a variety of different 
variations analagous to the family of B-trees. In Section~\ref{sec:rtrees}
we outline the original R-tree paper, and in Section~\ref{sec:variants}
we examine the variants and draw appropriate comparisons.

\subsection{R-Trees}
\label{sec:rtrees}
In 1984, Guttman first proposed the idea of modifying the B-tree structure to
use minimum bounding rectangles (MBR) as a way to restrict the search space 
during a lookup for spatial data. This data structure is called the R-tree.
R-trees are structured similarly to B-trees except, instead of having separation
values in each internal node that divide its subtrees, R-tree internal node
entries correspond to MBRs that bound its descendents. For instance, the MBR of 
a particular node completely overlaps the MBRs of the nodes of its child and 
its child's children. ike in the B-tree case, nodes correspond to disk pages 
and leaves point to database objects.

R-trees are bound by two parameters $m$ and $M$, the minimum and maximum number
of entries for each node except the root, respectively. An internal node entry 
is of the form ($mbr$, $p$), where $mbr$ is the MBR containing the MBRs of its 
descendents and $p$ is the pointer to its child subtree. The $mbr$ entry is of 
the form ($I_{0}$, $I_{1}$, ..., $I_{n-1}$), where $n$ is the number of 
dimensions and $I_{i}$ is of form $[a$,$b]$, a closed bounded interval along 
the i-th dimension. Similarly, a leaf node entry is of the form ($mbr$, $oid$), 
where $mbr$ is the MBR containing the object, and oid is the identifier for the 
object in the database. Finally, the root node must have at least three entries
except if it is a leaf.

There are multiple operations that are associated with an R-tree, which we 
discuss in the following sections.

%% Would be nice to have a picture illustrating the R-Tree
\subsubsection{Search and Update}
In order to find all entries overlapped by a bounding rectangle $S$ in the 
R-tree, the pseudocode of Figure~\ref{fig:R_Tree_Search} is used. In a similar 
fashion to a B-tree traversal, each node in the tree starting from the root is 
checked for overlap using the $mbr$ field in the entry. If there is overlap, 
the search descends into the subtree pointed to by $p$ until it reaches a leaf. 
If the leaf entry's $mbr$ overlaps with $S$, then the object ID $oid$ is 
returned. Note that there is no worst-case performance guarantee for this
algorithm.

Updates are slightly more complex. If a tuple is changed such that the MBR
covering it is also changed, its record in the R-tree must be deleted and then
reinserted. This makes the cost of an update fairly expensive.

\begin{figure}[t]
\begin{algorithmic}
	\Function{Search}{$T$, $S$}
		\If{$T$ is not a leaf}
			\ForAll{$E$ in $T$}
				\If{$E.mbr$ overlaps $S$}
					\State \Call{Search}{$E.p$, $S$}
				\EndIf
			\EndFor
		\Else
			\ForAll{$E$ in $T$}
				\If{$E.mbr$ overlaps $S$}
					\Return $E.oid$
				\EndIf
			\EndFor
		\EndIf
	\EndFunction
\end{algorithmic}
\caption{Pseudocode for finding all entries in a R-tree rooted at T overlapped by a search rectangle S}
\label{fig:R_Tree_Search}
\end{figure}

\subsubsection{Insert}
Insertion again is similar to B-tree insertion methods, as illustrated by the 
algorithm in Figure~\ref{fig:R_Tree_Insert}. The algorithm traverses the tree 
to find the appropriate node to insert into and performs splits when inserting
into full nodes, but there is one important distinction: node splitting 
heuristics. B-tree node splitting is simple since it is only necessary to 
partition the two resulting nodes into two equally sized nodes. In R-trees, the 
goal typically is to create two nodes such that it is unlikely for both to be
examined on subsequent searches by minimizing the total area of the MBRs for
both nodes.

In the original R-tree paper, Guttman discusses three different types of node 
splitting algorithms of different complexities: linear, quadratic, and 
exponential. The linear split algorithm is the most lightweight, but may
result in suboptimal splits since it does not perform an exhaustive search on
all possible groupings. It divides a group of entries into two groups by first 
selecting two seed entries to be the first entries in each group. The two 
entries with the highest normalized separation along any dimension are chosen.
Of the remaining entries, a random one is chosen and placed in the group that 
would have the least enlargement. The exception is if the number of remaining 
entries plus the number of entries in one group is equal to $m$, the minimum 
number of entries in a node, all remaining entries are put into that group. 

The quadratic split algorithm is identical to the linear algorithm, except the 
entries chosen to be seeds are the ones with that have the most wasted space
if grouped together; in other words, the two entries that have the maximum $d$
where $d = area(MBR_{i,j}) - area(E_{i}) - area(E_{j}) $.
Also, the way the remaining entries are added is different than the linear 
algorithm. In this case, the area increase for both groups is calculated for each
entry, and the entry with the maximum difference between the two will be inserted 
into the group that will result in less enlargement.

The exponential split algorithm is an exhaustive search that enumerates all possible
groupings for the entries. Given an R-tree with $M$, the maximum number of entries
in a node, the search space for this algorithm is on the order of $2^{M-1}$.

\begin{figure}[t]
\begin{algorithmic}
	\Function{Insert}{$T$, $E$}
		\State \Call{ChooseLeaf}{$T$, $E$}
		\Comment{Traverse tree from $T$ to appropriate leaf.
			At each level choose node $L$ whose MBR will
			need the least enlargement to cover E.mbr or
			if there is a tie, choose node with minimum
			area. Return $L$
		}
		\If{$L$ is not full}
			\State Insert $E$ into $L$
		\Else
			\State \Call{SplitNode}{$L$}
			\Comment{Returns $L$ and $LL$ containing $E$ and the old
				entries of $L$}
		\EndIf
		\State \Call{AdjustTree}{$L$}
		\Comment{Ascend from leaf node $L$ up to the root $T$
			and propagate splits. }
	\EndFunction
\end{algorithmic}
\caption{Pseudocode for inserting into an R-tree rooted at T given an entry E}
\label{fig:R_Tree_Insert}
\end{figure}

\subsubsection{Delete}
Deletion is handled by the algorithm of Figure~\ref{fig:R_Tree_Delete}. First,
we find the leaf containing the entry to be deleted. Then, we handle the case
where nodes are underfull by calling \emph{CondenseTree} on the leaf that held
the entry. Instead of merging the underfull node with a sibling like in a 
B-tree, the node is deleted, the other nodes in the leaf are reinserted into 
the R-tree, and the ancestor MBRs are adjusted accordingly. Guttman argues that
reinsertion has two advantages; first, it is easier to implement, and second, 
it prevents deterioration of the R-tree. We will see in other implementations
of the R-tree that this is not the only strategy for deletion.

\begin{figure}[t]
\begin{algorithmic}
\Function{Delete}{$T$, $E$}
	\State \Call{FindLeaf}{$T$, $E$}
	\Comment{Traverse tree from $T$ to appropriate leaf.
			Return node $L$ containing $E$}
	\State Delete $E$ from $L$
	\State \Call{CondenseTree}{$L$}
	\Comment{Given leaf $L$ where $E$ was deleted, if $L$ was
	underfull, reinsert other entries in $L$, delete, and propagate changes
	upward}
\EndFunction
\end{algorithmic}
\caption{Pseudocode for deleting an entry E from an R-tree rooted at T}
\label{fig:R_Tree_Delete}
\end{figure}

\subsection{R-Tree Variants}
\label{sec:variants}
At the time, the R-tree  was a great solution to the problem of efficiently searching
for spatial data in a database; however, there were certain issues with the original
implementation. For instance, the complexity of the insertion algorithm
depends on the complexity of the node splitting function, which is typically
quadratic. Similarly, the heuristics for node splitting affects the efficiency of 
the search algorithm. These challenges are discussed in detail in 
Section~\ref{sec:impchal}. In the below sections, we discuss the main R-Tree
variants that seek to solve some of the disadvantages of the R-tree.

\subsubsection{$R^{+}$-Trees}
Shortly after Guttman's paper in 1984, Sellis, Roussopoulos, and Faloutsos proposed 
the $R^{+}$-tree. The problem they sought to solve was the performance degredation 
during a range search due to high MBR overlap. Basically, the $R^{+}$-tree tries to 
reduce the number of paths explored (and consequently the number of I/O operations) 
during a search by disallowing overlap in MBRs in internal nodes of the same level. 
Figure~\ref{fig:R+-Tree} illustrates this concept. Compare this to the R-tree of 
Figure~\ref{fig:R-Tree}, which allows overlap. As we can see, objects that span 
across multiple MBRs are stored in multiple leaves. This duplication has several 
consequences that affect searches, deletion, and insertion.

% Search & Delete
Searches in $R^{+}$-trees are performed the same as in the R-tree case, except
redundant entries must be eliminated during a range query. However, in point queries
$R^{+}$-trees are guaranteed to only traverse one path since no MBRs overlap in 
internal nodes. Conversely, deletion is slightly more complex than in the R-tree case
since there may be more than one entry that must be deleted. This means that multiple 
leaves may be visited. Essentially, the $Delete$ algorithm traverses the tree using
the MBR of the entry to delete as the search parameter and when it reaches a leaf, it 
removes the corresponding entry and propagates the change upward. In \cite{SellisRoussopoulosFaloutsos86} the authors do not discuss how they handle underfull nodes.


% Insertion
Insertion is slightly different than the R-tree. Since MBRs do not overlap, an
object may be added to more than one leaf node, which is not the case in an R-tree. 
Thus, the insertion algorithm in Figure~\ref{fig:R+_Tree_Insert} is used. We see that
the function recursively inserts $E$ into all leaves with overlapping MBRs instead of 
using path traversal to select just one. Note that the algorithm $SplitNode$ is 
also different from the R-tree algorithm due to the disjoint MBR requirement because
changes may require downward in addition to upward propagation. 

More specifically, the $SplitNode$ algorithm takes care to partition the entries in 
the node into two new nodes with non-overlapping MBRs and propagates this change 
downward and upward, if necessary, using recursive calls to $SplitNode$ on the node's
children and parents, respectively. For brevity, we do not include the $SplitNode$ 
algorithm, but we discuss the $Partition$ algorithm of \ref{fig:R+_Tree_Partition} 
used to determine the regions described by the new nodes in more detail. Basically, 
costs for cuts in each dimension are calculated using $Sweep$. $Sweep$ calculates the
cost based on resultant dead space and number of rectangle splits. The cut that has 
the smallest cost is chosen and the entries in the node to be split are placed 
according to their MBRs. 

\begin{figure}
\begin{algorithmic}
	\Function{Insert}{$R$, $E$}
		\If{$R$ is not a leaf}
			\ForAll{Entries $X$ in $R$}
				\If{$X.mbr$ overlaps $E.mbr$}
					\State \Call{Insert}{$X.p$, $E$}
				\EndIf
			\EndFor
		\Else
			\If {$R$ is full}
				\State \Call{SplitNode}{$R$}
			\Else
			\EndIf
		\EndIf
	\EndFunction
\end{algorithmic}
\caption{Pseudocode for inserting an entry E given a $R^{+}$-tree rooted at R}
\label{fig:R+_Tree_Insert}
\end{figure}

\begin{figure}
\begin{algorithmic}
	\Function{Partition}{$S$, $ff$}
		\If{No partition required}
			\State {$R \Leftarrow$ Node containing $S$}
			\Return {($R$, empty)}
		\EndIf
		\State $O_{x} \Leftarrow$ minimum x coordinate of all r in $S$
		\State $O_{y} \Leftarrow$ minimum y coordinate of all r in $S$
		\State $(C_{x}, x_{cut}) \Leftarrow$ \Call {Sweep}{"x", $O_{x}$, $ff$}
		\State $(C_{y}, y_{cut}) \Leftarrow$ \Call {Sweep}{"y", $O_{y}$, $ff$}
		\Comment{Starting from $O_{x}$ or $O_{y}$ on its respective axis, pick 
		the first $ff$ rectangles sorted on the input axis. Return the 
		cost to split along this axis and the maximum value of the cut}
		\State $cut \Leftarrow min(Cx, Cy)$
		\State {Distribute $S$ according to cut}
		\State $R \Leftarrow$ Node containing entries in first subregion of cut
		\State $S' \Leftarrow$ Set of rectangles not in $R$ 
		\Return ($R$, $S'$)
	\EndFunction
\end{algorithmic}
\caption{Pseudocode for partitioning a set of rectangles S and fill factor ff into
	a node $N$ containing the rectangles of the first subregion and set $S'$ of
	the remaining rectangles}
\label{fig:R+_Tree_Partition}
\end{figure}

% Packing
%\cite{SellisRoussopoulosFaloutsos86} also describes a packing algorithm....

% Performance comparison
Compared to R-trees, $R^{+}$-trees have better search performance in certain cases,
such as when there are many small objects and a few large objects or during point 
queries, according to the analysis performed in \cite{DBLP:conf/vldb/SellisRF87}. 
However, there are also cases in which $R^{+}$-trees have worse search performance. 
For instance, when there are many large objects in the database that span across many
MBRs, one object may be replicated across many different nodes which causes range
queries to be less efficient.

\subsubsection{$R^{*}$-Trees}

% Search, Insert, Node Splitting, Delete


% Not sure this fits here.
%\subsubsection{Hilbert R-Tree}

