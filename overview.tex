% Overview 
To solve the problem of performing efficient searchs on spatial data, 
Guttman proposed the R-tree, which inspired a variety of different 
variations analagous to the family of B-trees. In Section~\ref{sec:rtrees}
we outline the original R-tree paper, and in Section~\ref{sec:variants}
we examine the variants and draw appropriate comparisons.

\subsection{R-Trees}
\label{sec:rtrees}
In 1984, Guttman first proposed the idea of modifying the B-tree structure to
use minimum bounding rectangles (MBR) as a way to restrict the search space 
during a lookup for spatial data. This data structure is called the R-tree.
R-trees are structured similarly to B-trees except, instead of having separation
values in each internal node that divide its subtrees, R-tree internal node
entries correspond to MBRs that bound its descendents. For instance, the MBR of 
a particular node completely overlaps the MBRs of the nodes of its child and 
its child's children. ike in the B-tree case, nodes correspond to disk pages 
and leaves point to database objects.

R-trees are bound by two parameters $m$ and $M$, the minimum and maximum number
of entries for each node except the root, respectively. An internal node entry 
is of the form ($mbr$, $p$), where $mbr$ is the MBR containing the MBRs of its 
descendents and $p$ is the pointer to its child subtree. The $mbr$ entry is of 
the form ($I_{0}$, $I_{1}$, ..., $I_{n-1}$), where $n$ is the number of 
dimensions and $I_{i}$ is of form $[a$,$b]$, a closed bounded interval along 
the i-th dimension. Similarly, a leaf node entry is of the form ($mbr$, $oid$), 
where $mbr$ is the MBR containing the object, and oid is the identifier for the 
object in the database. Finally, the root node must have at least three entries
except if it is a leaf.

%% Would be nice to have a picture illustrating the R-Tree
\subsubsection{Search}
In order to find all entries contained by a bounding rectangle in the R-tree, 
the pseudocode of Figure~\ref{fig:R_Tree_Search} is used.

\begin{figure}
\begin{algorithmic}
	\Function{Search}{$T$, $S$}
		\Comment {Return all entries contained by S given an R-tree 
			rooted at T}
		\If{$T$ is not a leaf}
			\ForAll{$E$ in $T$}
				\If{$E.mbr$ overlaps $S$}
					\State \Call{Search}{$E.p$, $S$}
				\EndIf
			\EndFor
		\Else
			\ForAll{$E$ in $T$}
				\If{$E.mbr$ overlaps $S$}
					\Return $E.oid$
				\EndIf
			\EndFor
		\EndIf
	\EndFunction
\end{algorithmic}
\caption{Pseudocode for searching a R-tree given a search rectangle}
\label{fig:R_Tree_Search}
\end{figure}

\subsubsection{Insert}
Insert.
\subsubsection{Delete}
Delete

\subsection{R-Tree Variants}
\label{sec:variants}
Much like its cousin, the B-tree, the R-tree has a few main variants such as
the R$^{+}$-tree and the R$^{*}$-tree, which we discuss in the following sections.

\subsubsection{R+-Trees}
\subsubsection{R*-Trees}

% Not sure this fits here.
%\subsubsection{Hilbert R-Tree}

