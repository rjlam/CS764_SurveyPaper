% Overview 
To solve the problem of performing efficient searchs on spatial data, 
Guttman proposed the R-tree, which inspired a variety of different 
variations analagous to the family of B-trees. In Section~\ref{sec:rtrees}
we outline the original R-tree paper, and in Section~\ref{sec:variants}
we examine the variants and draw appropriate comparisons.

\subsection{R-Trees}
\label{sec:rtrees}
In 1984, Guttman first proposed the idea of using minimum bounding rectangles
(MBR) as a way to restrict the search space during a lookup for spatial data.


\subsubsection{Search}

\subsubsection{Insert}

\subsubsection{Delete}


\subsection{R-Tree Variants}
\label{sec:variants}
Much like its cousin, the B-tree, the R-tree has a few main variants such as
the R$^{+}$-tree and the R$^{*}$-tree, which we discuss in the following sections.

\subsubsection{R+-Trees}
\subsubsection{R*-Trees}

% Not sure this fits here.
%\subsubsection{Hilbert R-Tree}

