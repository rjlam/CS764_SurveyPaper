\rbase-trees were introduced as a purely heuristic-based data structure for satisfying range queries.
Threir success for that and many other spatial queries has prompted intense study of why it works.
This has in turn improved our understanding of spatial data.
More immediately, a veritable zoo of R-tree variants and improvements exist, both testament to the value of the underlying idea of R-trees as well as the great need for spatial queries.
By presenting the core R-tree, the attempts at improving and understanding them, and the many variants since developed, we hope to provide a starting point to exploring this exciting topic.

