\documentclass[10pt, twocolumn]{article}

%Packages
\usepackage{fullpage}
\usepackage{algpseudocode}
\usepackage{hyperref, url}
\usepackage{cite}
\usepackage{graphicx}
\usepackage{caption, subcaption}
\usepackage{comment}
\usepackage{multirow}
\usepackage{booktabs}
\usepackage{times}

\usepackage{amsmath}
\usepackage{amssymb}
\usepackage{complexity}

%\usepackage{fullpage}

\title{A Survey of R-Trees}
\author{
	Aaron Gorenstein\\
	University of Wisconsin - Madison\\
	\texttt{agorenst@cs.wisc.edu}
	\and
	Rebecca Lam\\
	University of Wisconsin - Madison\\
	\texttt{rjlam@cs.wisc.edu}
	\and
	Cathrin Weiss\\
	\texttt{cweiss@cs.wisc.edu}
}

\begin{document}

\newcommand{\rbase}{R}
\newcommand{\rstar}{$\text{R}^*$}
\newcommand{\rplus}{$\text{R}^+$}

\maketitle
\thispagestyle{empty}

\input{abstract}

\section{Introduction}
\label{sec:intro}
Although they are by no means the methods for accessing spatial data, the \rbase-tree and 
its plethora of variants seem to be ubiquitous in spatial databases. They are for instance 
used in scientific and medical research, geographic information systems, spatial datamining, 
computer-aided design, and commercial systems. Other spatial access methods include 
quadtrees, k-d-B trees, and multidimensional hashing\cite{samet95, gaedegunther98}.
However, we focus our survey on \rbase-trees and relevant topics such as the details of 
the \rbase-tree's basic algorithms and some of its implementation issues. We also discuss 
how \rbase-trees are used in spatial queries, optimization of \rbase-trees, and the 
\rbase-tree cost model. Given its prevalence in spatial databases, we conclude our paper with 
an overview of \rbase-trees in parallel systems, spatio-temporal databases, and other applications. 

Other work that discuss \rbase-trees include surveys on multidimensional access methods
by Gaede and G\"{u}nther \cite{gaedegunther98} and Samet\cite{samet95}. These papers
discuss \rbase-trees and their variants only on a perfunctory level since the scope of their 
work encompasses all types of spatial access methods. A fairly exhaustive book on 
\rbase-trees and its variants has been previously published by Manolopoulos et al.
in \cite{thebook} that covers most of the work on \rbase-trees before 2006. Our survey 
extends these works with an in-depth discussion on more recent \rbase-tree variants as 
well as newer optimizations and solutions to \rbase-tree implementation.

The rest of the paper is as follows. In Section~\ref{sec:overview} we discuss the
original \rbase-tree and it main variant, the \rstar-tree. In Section~\ref{sec:impchal}
we examine basic queries using R-trees and associated optimizations. In 
Section~\ref{sec:dbchal} we talk about topics related to \rbase-tree performance, and 
finally in Section~\ref{sec:apps} we extend our discussion to the realm of parallel and
distributed systems, as well as various real-world applications of \rbase-trees.




\section{Overview of R-Trees}
\label{sec:overview}
% Overview 
To solve the problem of performing efficient searchs on spatial data, 
Guttman proposed the R-tree, which inspired a variety of different 
variations analagous to the family of B-trees. In Section~\ref{sec:rtrees}
we outline the original R-tree paper, and in Section~\ref{sec:variants}
we examine the variants and draw appropriate comparisons.

\subsection{R-Trees}
\label{sec:rtrees}
In 1984, Guttman first proposed the idea of modifying the B-tree structure to
use minimum bounding rectangles (MBR) as a way to restrict the search space 
during a lookup for spatial data. This data structure is called the R-tree.
R-trees are structured similarly to B-trees except, instead of having separation
values in each internal node that divide its subtrees, R-tree internal node
entries correspond to MBRs that bound its descendents. For instance, the MBR of 
a particular node completely overlaps the MBRs of the nodes of its child and 
its child's children. ike in the B-tree case, nodes correspond to disk pages 
and leaves point to database objects.

R-trees are bound by two parameters $m$ and $M$, the minimum and maximum number
of entries for each node except the root, respectively. An internal node entry 
is of the form ($mbr$, $p$), where $mbr$ is the MBR containing the MBRs of its 
descendents and $p$ is the pointer to its child subtree. The $mbr$ entry is of 
the form ($I_{0}$, $I_{1}$, ..., $I_{n-1}$), where $n$ is the number of 
dimensions and $I_{i}$ is of form $[a$,$b]$, a closed bounded interval along 
the i-th dimension. Similarly, a leaf node entry is of the form ($mbr$, $oid$), 
where $mbr$ is the MBR containing the object, and oid is the identifier for the 
object in the database. Finally, the root node must have at least three entries
except if it is a leaf.

%% Would be nice to have a picture illustrating the R-Tree
\subsubsection{Search}
In order to find all entries contained by a bounding rectangle in the R-tree, 
the pseudocode of Figure~\ref{fig:R_Tree_Search} is used.

\begin{figure}
\begin{algorithmic}
	\Function{Search}{$T$, $S$}
		\Comment {Return all entries contained by S given an R-tree 
			rooted at T}
		\If{$T$ is not a leaf}
			\ForAll{$E$ in $T$}
				\If{$E.mbr$ overlaps $S$}
					\State \Call{Search}{$E.p$, $S$}
				\EndIf
			\EndFor
		\Else
			\ForAll{$E$ in $T$}
				\If{$E.mbr$ overlaps $S$}
					\Return $E.oid$
				\EndIf
			\EndFor
		\EndIf
	\EndFunction
\end{algorithmic}
\caption{Pseudocode for searching a R-tree given a search rectangle}
\label{fig:R_Tree_Search}
\end{figure}

\subsubsection{Insert}
Insert.
\subsubsection{Delete}
Delete

\subsection{R-Tree Variants}
\label{sec:variants}
Much like its cousin, the B-tree, the R-tree has a few main variants such as
the R$^{+}$-tree and the R$^{*}$-tree, which we discuss in the following sections.

\subsubsection{R+-Trees}
\subsubsection{R*-Trees}

% Not sure this fits here.
%\subsubsection{Hilbert R-Tree}



\section{Basic Queries and R-tree Optimizations}
\label{sec:impchal}
\newcommand{\keyword}[1]{\textbf{#1}}

[It is incredibly important we cite \cite{thebook} properly throughout here: it has directed us to many good papers and starting points, and influenced the organization of this section heavily.]

The R-tree was introduced with the search, or \keyword{range} query.
Soon it was being applied to a variety of other queries.
Its apparent universality for spatial queries is natural motivation for improving R-tree performance.
In this section we discuss various advances beyond Guttman's original design.
Along the way we hope to imply some intuition about the underlying theory for R-tree performance.
In the next section we show what formal theory and considerations for R-trees have so far been developed.

Before delving into the improvements for R-trees it is important to provide some impression of the broad context in which we find R-trees.
Namely, what are the spatial queries R-trees are used for?
How are these queries efficiently satisfied using the structure of the R-tree?
As it happens there is an inordinate number of queries for which R-trees may be used, and we cannot hope to catalog them all here.
One broad classification scheme groups queries into three types.
\keyword{Topological} queries are based on basic geometric properties of the data.
The quintessential \keyword{range} query falls into this category.
\keyword{Directional} queries involve predicates of the form ``above'', or ``east''---filtering according to global positions.
An example of this is skyline, or dominating points computations.
Lastly, \keyword{distance} queries consider distance between data elements.
The exemplar here is nearest-neighbor computation.
Within these three categories there exists a great many queries.
Even accounting for the different types of topological relations between MBRs is a daunting task \cite{papadiassellistheodoridisegenhofer95}.
Also note that these categories are not a neat partition: would the ``nearest surrounder'' query be directional or distance?

In light of this overwhelming diversity we will consider just three queries and walk through their execution on an R-tree.
The goals of this in-depth examination are three:
provide a context for the reader to keep in mind as we discuss R-tree improvements;
demonstrate how R-trees can service very different queries;
demonstrate how---despite their differences---these varied queries share similar bottlenecks in the R-tree, implying general optimizations are possible.
The three queries are \keyword{range}, \keyword{spatial join}, and \keyword{nearest neighbor}.

\subsection{The Range Query}
The range query is perhaps the simplest spatial query and remains essentially unchanged since its introduction \cite{guttman84}.
The idea is this: given a database of $R$ spatial objects represented by their MBRs and an input rectangle $s$, output all $r\in R$ such that $r\cap s\neq\emptyset$.
Described briefly in English: at sub-tree $T$, recurse at each child if that child's MBR has a nonempty intersection with $s$.
If $T$ is a leaf, output $T$ if it has a nonempty intersection with $s$.
There exist many variations---perhaps the query is only for objects strictly containing, or contained within, $s$ \cite{gaedegunther98}.
This general query's popularity has warranted intense study, and extensive tests have compared how R-tree variants perform on different inputs \cite{papadiassellistheodoridisegenhofer95}.

We take the time here to emphasize that even in this fundamental query there exist complications beyond just the various forms the query can take.
Some thought reveals that a range query can result in output equal to $\emptyset$ or to the whole tree.
Moreover, this is true even if the range is the degenerate case: the point.
A surprising fact is that there exist R-trees and queries such that the output is $\emptyset$ but the \emph{whole tree must be traversed}.
This is true not just for Guttman's initial R-tree, but many advanced types as well (see Theorem 3 of \cite{argeberghaverkortyi04}).
As testament to its utility, we note that the range query on R-trees was adapted to \emph{approximating similarities between frequencies} \cite{agrawalfaloutsosswami93} by mapping the $k$ major features of the DFT transform into $k$-dimensional space.

\subsection{The Spatial Join}
A spatial join is nothing more than a general join, just with a spatial predicate---namely, join $r_1$ with $r_2$ iff $r_1\cap r_2\neq\emptyset$ \cite{brinkhoffkriegelseeger93}.
This is a more general operation than range query---that is, all variations of range queries are easily reducible to the appropriate variant of a spatial join \cite{gaedegunther98}.
The core spatial join algorithm \cite{brinkhoffkriegelseeger93} follows the same principles as the range query.
It assumes that the two trees, $R_1$ and $R_2$, are the same height.
For a given child $n_1\in R_1$ and child $n_2\in R_2$, if $n_1\cap n_2=\emptyset$ it then moves on to the next child in $R_2$.
Otherwise, we recurse in a DFS fashion.
Observe that if the MBR are better-clustered in some fashion, we can conclude that $n_1\cap n_2=\emptyset$ at a higher level in the tree, thus saving page accesses.
There exists a variation of spatial join over R-trees which proceeds in a breadth-first-search fashion \cite{huangjingrundensteiner97}.
This is motivated by the idea that understanding how the join behaves at level $k$ might better guide the iteration over the pages in level $k+1$.
The underlying motivation for this alternative is that it is a hard problem to figure out which nodes we ``really'' want to explore, given only the MBRs.

See also \cite{papadopoulosrigauxscholl99} for evaluating different spatial joins on varios R-tree variants.
See also \cite{vassilakopouloscorralkaranikolas11} for join within a single R-tree, self-joins.
Find a paper (there are many) that say join reasoning from B-trees don't naturally extend to R-trees.
Huge survey \cite{jacoxsamet07}.

\subsection{The Nearest Neighbor Query}
The nearest neighbor (NN) query takes as input a point $p$ and outputs the $k$ nearest objects to it (usually under Euclidean distance) \cite{roussopouloskelleyvincent95}.
The algorithm for NN on R-trees differs from range or join in part because it is not merely doing set-theoretic operations on the data. 
Rather, at each level of the tree we compute MINDIST, the minimum distance from $p$ to \emph{some} object in the MBR $R$ for each $R$ at that level.
This naturally leads to a DFS traversal: at level $k$ we computer the MINDIST to each node, sort them in that order, and recurse on the first one.
Some reasoning about the properties of MINDIST allows for safe pruning of these lists, allowing us to skip many nodes.
The heuristic of sorting by MINDIST is not optimal---rather, it is ``optimistic'' in the sense that if the MBRs really do reflect the nearest point, it will terminate quite quickly.
There exist other heuristics as well in the same paper \cite{roussopouloskelleyvincent95}.

See also \cite{corralalmendros-jimenez07}, comparisons between R-tree performance on various distance queries.

\subsection{Optimization Intuition}
From these three queries, the underlying cause of inefficiency is obvious: that for any non-leaf node, the MBR does not give us all the information we need to know if we should explore it or not.
If we are wasteful during a range query, it is because our inner node MBRs are so big-but-sparse that we have to search them despite their children's empty intersection.
We can be wasteful during a join for much the same reason.
In a nearest neighbor search, if the inner MBRs do not provide a good approximation of their children, then the MINDIST values would fail to allow much pruning in our branch-and-bound.
Informally speaking, it seems that all these queries suffer if the inner MBRs cover much more area than their children, or the inner MBRs have a much greater degree of overlap with each other than the actual data (leaves).
So in our efforts to improve our R-tree performance, we want to somehow grow a ``better-organized tree''.
Implicit in this desire is an important difference between B-trees and R-trees.
Consider a set of data $\mathcal S$, an index $I$ implying an order-of-insertion for those elements $s\in \mathcal S$, and the B-tree $B_I$, generated by inserting $s_{i+1}$ into $B_I$ after $s_{i}$.
Whatever $I$ we choose, by the linear nature of $\mathcal S$, the resulting $B_I$ will not differ ``too much'' from any $B_{I\prime}$.
Obviously, the leaves always end up in the same order.
However, this is \emph{not} the case at all for R-trees.
Given $I$ and $I\prime$, the R-trees $R_I$ and $R_{I\prime}$ may wildly differ in terms of the inner nodes created.
Thus our idea of growing an R-tree in a ``smarter'' fashion has real meaning.

\paragraph{Node Splitting}
As a basic R-tree's construction (and hence performance) is somewhat at the mercy of its insertion order, a natural first step to improving R-trees is to take the time to split nodes so that the tree maintains a good structure.
When faced with the task of splitting a node in a B-tree, the split is obvious.
When splitting an R-tree node, however, we have the freedom and burden to choose among $\binom{M}{m}$ possible partitions of $M$ data points into 2 sets of size $m=\frac{M}{2}$.
In conceiving the R-tree Guttman realized the importance of a ``good'' split and proposed three algorithms for it.
The initial \cite{guttman84} followed by the heuristic improvement in R* \cite{beckmannkriegelschneiderseeger90} followed by optimal \cite{garcialopezleutenegger98}.
This isn't the end-all be-all, as 

More persistent tree improvement (how dose this mesh with ``applying static advancements''?).
See \cite{leehsujensencuiteo03} for more tree reorg.

Simplifying splits: have a strict ordering, see \cite{kamelfaloutsos94}.


\paragraph{Static R-Tree Construction}
In our concern for computing a good split as the R-tree recieves new data, we are naturally tempted to imagine the situation where the data is unchanging, static.
This is far from contrived: the 1930s US census data is unlikely to change.
The paper on bulk loading \cite{garcialopezleutenegger98a}.
Index optimization \cite{gavrila94}.
Hilbert R-tree \cite{kamelfaloutsos94}.

\paragraph{Optimizing Dynamic Space Usage}
In our considering static R-trees, we have seen the utility of dense packing.
Intuitively, if the tree is dense, there are fewer nodes (perhaps even fewer layers), and so we have less tree to check.
This gives us motivation for improving space usage in dynamic R-trees: not to \%100 of course, but in read-heavy loads B-trees benefit from more density, so it is natural to be curious if we can make the same improvements for R-trees.
Improving spatial usage, dense R-trees.
The only citation from \cite{thebook} is \cite{huanglinlin01}.
I bet more have arrived.
Dynamic Hilbert R-tree from \cite{kamelfaloutsos94}, the classic.

\paragraph{Applying Static Advancements to Dynamic R-Trees}
Grafting is a cool advance, but seemingly ignored \cite{schrekchen00}.
Observe that R* has some satic improvements \cite{beckmannkriegelschneiderseeger90}.
Is there a quintessential bulk loading-in-dynamic trees or lazy paper?
Perhaps this covers both, to start with: \cite{argehinrichsvahrenholdvitter99}.
This introduces the LUR, the lazy update R-tree \cite{kwonleelee02}.

\paragraph{Avoiding Overlapping MBRs}
A fundamental inefficiency with the R-tree---especially for range queries---is that we must sometimes guess which subtree on which to recurse, thereby often computing ``dead-ends''.
This guessing occurs when our query overlaps the MBR of our R-tree.
This is unavoidable, but the likelihood of this query overlap is minimized when the overlap among our MBRs is minimized.
This motivates the definition of the \emph{stabbing number} \cite{berggudmundssonhammarovermars00}.
Consider the MBRs of a single layer of the R-tree (the leaves constitute a layer).
The stabbing number is the maximum number of rectangles containing a single point---querying that point ``stabs'' the greatest number of rectangles possible.
The authors \cite{berggudmundssonhammarovermars00} use this notion to develop some formal lower bounds on R-tree queries.

Since its conception, substantive work has gone into determining the feasibility of computing optimal partitionings---i.e., MBRs with minimal stabbing number.
We can formalize our desire as computing a $r$-partition of our $N$ rectangles such that our $r$ new MBRs have \emph{minimal stabbing number}.
As it turns out, computing such a partition is $\NP$-hard; on the other hand, it is parameterized by $k$, which quickly becomes small as we consider each layer of the R-tree \cite{bergkhosraviverdonschotweele11}.
The algorithm was ultimately considered too slow, but perhaps future work will make it practical.
[Other work also exists on this, modern stuff too.]

Prior to this formal approach, the R+ tree \cite{sellisroussopoulosfaloutsos87} was the first to attempt to minimize stabbing number.
They changed the R-tree's design such that no MBRs on a single, inner layer overlapped with their siblings.
(Obviously we can only control the inner layers---the leaves may overlap.)
This design substantively deviates from ``normal'' R-tree implementations in that leaves are replicated, if they are contained in the MBR of multiple inner nodes.
This facilitates fast \emph{range} queries, but complicates other queries \cite{some paper talks about how it breaks down on perimiter queries or something---gaedegunther?samet?}.


\paragraph{Provable Optimality}
Basically, talk briefly about the optimal R-tree.
Or maybe this should be a lead-in to the formal discussion.


\section{Understanding R-tree Performance}
\label{sec:dbchal}
As we have toured the classic enhancements and improvements to R-tree performance, we have encountered experimental data hinting towards when and how R-trees perform well and why.
In this section we discuss the papers that took that information and worked towards a theory of R-trees.

\subsection{Cost model}
We hope to answer a very simple question: given a query, can the R-tree estimate how long that query will take?
The efficacy of an R-tree query is usually measured by the number of expected page accesses \cite{.....}.
It is important to note that with spatial data, it is not unusual for computation time, not IO, to be the bottleneck \cite{.....}.
Indeed, we assume our actual data is simple, perhaps axis-aligned rectangles themselves---otherwise taking into account the geometric complexity of our actual data further complicates things \cite{aboulnaganaughton00}.
However, IO is the one universal non-trivial source of expense, so we focus on that.

So we hope to estimate how many page accesses an R-tree needs to satisfy a query for a given data set.
Whereas predicting B-tree performance is relatively straightforward \cite{....}, recall how R-tree performance varies greatly on how ``well'' it is organized.
Thus, it seems that predicting a query's cost is not just a function of the dataset, but also the structure of the R-tree built on it.
Indeed, the initial cost models for R-trees were index-dependent \cite{.....}.
% \cite{see:list:in:theodoridisstefanakissellis}.
Moreover, those models largely assumed \emph{uniformly distributed} input---quite an assumption, especially for spatial data \cite{...}.

Fortunately, our understanding has advanced to the point where index-independent, cost-models are available for non-uniform data.
They are still parametrized by the spatial query, of course, but we are able to assume an arbitrary ``good'' R-tree.
One of the first was by Theodoridis et. al. \cite{theodoridissellis96,theodoridisstefanakissellis00}, for \emph{range} and \emph{join} queries.
Their cost model was only a function of the number of elements in the dataset and the spatial density.
Their inducing a \emph{density surface}, like a histogram mapping partitions of the space to the local density, was their key advance to extending their cost model beyond uniform input.
An alternative approach was fractal dimensionality, a concept introduced by \cite{....}.
It was extended to [stuff] by \cite{...}.

Our previous metrics assumed, or hoped-for, some a priori knowledge about the data.
Failing that, we could attempt to estimate the properties of the data, which of course implies a means of maintaining that estimation.
This data is often applied for query optimization \cite{chaudhuri98}, and the histogram is one of the main tools for that \cite{poosalahaasioannidisshekita96}.

[Histograms maintained to improve R-tree performance]

R-trees can be used \emph{as} spatial histograms \cite{achakeevseeger12}, and serves to estimate a cost of a \emph{range} query.
Their spatial histogram is to partition the data rectangles $r_1,\ldots,r_N$ into buckets, each bucket maintaining the average $x,y$ lengths of its constituents $r_i,\ldots,r_j$ and the bucket's density.
Implied by its having a density, the bucket induces an MBR over its rectangles $r_i,\ldots,r_j$.
In in their paper \cite{achakeevseeger12}, the authors decide on an equi-depth histogram.
This is complicated by the fact that computing their optimal partition is $\NP$-hard \cite{muthukrishnanpoosalasuel99}.
Their design, guided by the cost-model of \cite{theodoridissellis96}, was thus computed with heuristics; the details are in \cite{achakeevseeger12a}.
 %
There also exist histograms for the purpose of estimating, say, a join \cite{aboulnaganaughton00} [other papers exist].
 %Also takes into account other stuff.
 %Some experimental models: \cite{aboulnaganaughton00}, \cite{anyangsivasubramaniam01}, \cite{achakeevseeger12,achakeevseeger12a} has good datasets discussion, see for ``data model''.

\subsection{Optimality in R-tree}
[Note: arge et. al. have been the leaders in formal analysis here.
Optimal R-tree \cite{argeberghaverkortyi04}.
Optimal R-tree, high-level \cite{yi12}.
Optimizing for minimal stabbing number \cite{bergkhosraviverdonschotweele11}.
Upper bound on a range query \cite{kanthsingh99}.
Big discussion: why doesn't this have a formal link to our ``design intuition'' of the previous section?
Limitations: this is optimality for range, hard to talk about for other queries.
Bulk-loading required.]


\section{Extensions and Applications}
\label{sec:apps}
The discussion thus far has focused on spatial data in the domain of a single
processor system with the R-tree stored all on one disk. We now extend the 
scope of our survey of R-trees into the topics of parallel systems, 
spatiotemporal databases, and applications of R-trees.

% Went for a weird title..... can be changed :P
\subsection{R-trees in the Parallel Universe}
Before we begin our overview of R-trees in a parallel environment, we first 
explain some basic definitions. There are three levels of resource sharing in
a parallel system: shared everything, shared disk, and shared nothing
\cite{thebook}. Shared everything means that all processors share all 
resources including disk and memory. Shared disk means that each processor 
has its own private memory but share disks. Unsurprisingly, shared nothing 
architectures have private memory and disk for each processor, which
communicate using some type of network. Thus the basic challenge for R-tree 
implementation in such systems boils down to exploiting parallelism to 
increase both CPU and I/O performance. In the following sections we discuss
R-trees in multidisk, single-cpu systems and in multiprocessor systems.

\subsection{R-trees in Multidisk Systems}
The architecture of a multidisk system consists of a single processor and 
multiple disks. The focus then in these types of systems is I/O parallelism
and how to partition data to maximize performance while maintaining good load
balance. Below are a few structures that fall under this category.

\subsubsection{Independent R-trees}
The independent R-tree method \cite{kamel1992parallel}\cite{thebook} uses separate R-tree structures for each disk.
Data is distributed using two approaches: data distribution and space 
partitioning. The former uses hashing or round-robin (RR) to assign disks to
R-tree entries. The second partitions the R-tree into sections such that 
child nodes reside on the same disk as their parents. The first approach 
has good data load distribution but poor locality, whereas the second approach
has good locality but could have worse throughput on large queries.

\subsubsection{Super-node R-tree}
In the super-node method \cite{kamel1992parallel}\cite{thebook} there is only
one R-tree structure for the system. Each node in the structure consists of
\emph{d} pages distributed across \emph{d} disks. When node is accessed, each
page of the node (one page per disk) is read in parallel. This has the 
advantage of good load balancing, but it has the disadvantage of
always touching every disk regardless of the query.

\subsubsection{Multiplexed R-tree}
\cite{kamel1992parallel} tackles the problem of data partitioning with the 
multiplexed R-tree (MX R-tree), which is essentially the same as the R-tree 
in \cite{guttman84} but distributes the nodes across different disks. 
The root node always resides in main memory, and all other nodes reside on 
disk. Cross-disk pointers are used to refer to child nodes residing in 
different disks. 

Disk assignment is performed with the conflicting criteria of
data balance, where the number of nodes are partitioned equally across disks, 
area balance, where the total area covered by each disk is approximately 
equal, and proximity, where close nodes are put on separate disks in order to
maximize throughput. \cite{kamel1992parallel} examines several placement 
heuristics but determines that proximity index (PI) has the best performance. 
The proximity index is essentially a measure of the probability that two MBRs
will be retrieved by the same query. Thus in the PI scheme nodes are assigned 
to disk with the proximity index with data balance as the tiebreaker criteria.

\cite{kamel1992parallel} determines that the MX R-tree is outperforms the two
other methods of node-to-disk distribution both in terms of query latency and
load.
 
\subsection{R-trees in Multiprocessor systems}
More recent research has focused on multiprocessor systems, especially in 
lieu of the growing popularity of distributed services. And development of 
multicpu processors. 
The multiprocessor system, unlike the multidisk system, consists of multiple 
processors. This could be in either a multicore machine in a shared-resource
environment or a distributed system with many machines in a shared-nothing 
environment. As before, there is a tradeoff between performance and load 
balancing; however, in this case scalability, availability, and consistency
are additional challenges. We examine several R-tree variants that exist in
this domain and also discuss R-trees in the context of parallel implementation
issues.

%Shared something R-trees
\subsubsection{Master R-trees}
The Master R-tree (M-Rtree)\cite{koudas1996declustering} is an R-tree variant 
for distributed systems.
In this structure, there are two different types of machines: \emph{master} and 
\emph{client}. Suppose there is some R-tree $X$ that we wish to represent using 
the M-Rtree. The master contains a R-tree structure that contains the non-leaf 
nodes of $X$, and the leaf nodes of $X$ reside in the clients as pages. The leaf 
level in the master instead has information on which client has the desired page has the form ($siteID$, $pageID$), where $siteID$ refers to the client on which
the data resides. Note that there is only one master, and it holds the root of 
the M-Rtree structure.
Thus a query must contact the master, which searches the portion of the M-Rtree 
in its memory. All clients in the resultant list (after all relevant master leaf 
entries have been touched) of relevant $siteID$ and $pageID$ pairs are then sent 
the query MBR and the $pageID$. Each client then fetches the page, checks for 
pertinent objects, and sends back the results to the master. 

In this scheme, parallel search processing occurs among the clients, but the 
master is inherently a source of bottleneck since all operations must be sent to
it.

%Shared nothing R trees
\subsubsection{Master-Client R-trees}
Another variant for the shared-nothing environment is the Master-Client R-tree 
(MC-Rtree) proposed in \cite{schnitzer1999master},  which is similar to the 
M-Rtree in the previous section except that instead of having a R-tree index 
structure residing only in the master, each client also has a subtree. As before,
the master holds only non-leaf nodes; however, the leaf level entries on the 
master are of the form ($mbr$, $siteID$). Moreover, the clients each have R-tree
structures that index the objects assigned to that site. This means that on a 
query it traverses the tree as usual, and as soon as a master leaf has been 
touched it stores the client $siteID$ in a list and sends the query MBR to the 
site, which processes the request. The master then waits for clients to respond
with any overlapping objects. 

In contrast to the M-Rtree, the MC-Rtree does not wait to send a request to
clients, and \cite{schnitzer1999master} finds that it has significantly better 
response time than the M-Rtree due to the reduced network contention and the 
reduced latency from issuing non-blocking requests to clients as soon as a leaf 
is reached rather than waiting until the whole tree has been searched. Also 
there is less time spent in the master tree is smaller since the clients hold 
more parts of the tree than the M-Rtree. However, this approach is also limited
by the fact that all requests must go through the master.

%Other trees from before 2005?

%dR-tree... see references in p2pr paper.

\subsubsection{P2PR-Tree}
% From 2005! cool!
In \cite{mondal2005p2pr} Mondal, Lifu, and Kitsuregawa examine extending the 
R-tree structure for use in peer-to-peer (P2P) systems. The challenge in P2P 
systems is not only the huge number of machines but heterogeneity among them as 
well. Thus \cite{mondal2005p2pr} proposes the P2PR-tree, which is a 
decentralized hierarchical tree structure built with scalability and availability
in mind. 

In the P2P environment, each peer (machine) has a unique \emph{peerID} and holds
an R-tree covered by a MBR (\emph{peerMBR}). The P2PR-tree is structured
such that the top two levels (level 0 and level 1) of the index are statically 
assigned and lower levels are dynamically assigned. Entries in level 0 (the root)
are called \emph{blocks} and entries in level 1 are called \emph{groups}. Each 
peer stores a copy of the root node and all nodes in the path from the root to 
the peerMBR. PeerMBRs that overlap with multiple group MBRs are inserted into 
both corresponding subtrees. Since the top two levels of the tree are static, 
this tree is not height-balanced\cite{mondal2005p2pr}.

When a query is issued, it may be handled by any peer. When a peer receives a 
query, it checks if its peerMBR overlaps with the query MBR. If so it performs 
a search on its local R-tree structure and completes. Otherwise, it checks the 
node at the current query level (starts at 0 and gets incremented every time it
is forwarded) and forwards the query to any peer in each overlapping entry (each
peer knows one or more peers in each block).

Compared to the MC-Rtree of \cite{schnitzer1999master}, the P2PR-tree performs
very well. \cite{mondal2005p2pr} finds that this decentralized technique 
dramatically decreases response time in comparison to the MC-Rtree since it does
not require each query to be processed by a master. However, the P2PR-tree can
still suffer from the effects of skewed data sets since ``hot'' nodes may have
higher response time if many requests are sent to it.

\subsubsection{SD-Rtree}
% from 2007! cooler!
The last multiprocessor R-tree variant we discuss is the SD-Rtree of 
\cite{du2007sd}. This structure is a binary tree residing on a set of servers.
Internal nodes maintain the ID of its parent and links to its two children of 
the form ($id$, $mbr$, $height$, $type$), where $id$ is the ID of the server 
storing that child, $height$ is the height of the child subtree, and $type$ 
indicates whether the child is a leaf or an internal node. Each leaf node stores 
the indexed objects, as usual.

\cite{du2007sd} circumvents the problem of congestion in upper level nodes by 
maintaining an \emph{image} which holds a snapshot of the tree and determines 
the server that most likely has the desired data. On receipt of a query, a 
server first checks its image and then forwards the query to the server it 
supposes holds the data. If it exists on the target server, then it returns the
desired data. Otherwise, the server examines the SD-Rtree from bottom up until a 
node containing the query MBR is found. %Similar to \cite{schniter1999master}, 
%each data node maintains the path from root to the node and uses 
%this information to forward the query when necessary. This also helps with 
%load balancing on the upper levels.

Both SD-Rtree and P2PR-tree are able to decentralize query handling and are able
to scale up well. P2PR-tree has the advantage of having a completely dynamic 
structure, but it does have additional memory cost due to having an image of the 
index structure.

% ACID?
\subsection{Parallel System Implementation Issues}
With the introduction of additional processes in a database there inherently
comes the issue of correctness. Operations may contend for the same data 
structure and without some method of concurrency control we have no guarantees
on the data in the database. We focus on the topics of query processing, 
concurrency control, recovery, and data migration. 

\begin{description}
	\item[Query Processing]
	\item[Concurrency Control]
	\item[Recovery]
	\item[Data Migration]
\end{description}

\subsection{R-trees in Spatiotemporal Databases}
There is extensive research for using R-tree variants in spatiotemporal
databases. Essentially spatiotemporal databases add time as an extra dimension
in addition to spatial data. This added dimensionality creates new challenges
[add more to this]. 

The R-tree variants used in spatiotemporal databases can be categorized based
on the time used for indexing data. Structures may index historical 
spatiotemporal data, current data, or future data. These are described more
in detail below.

\subsubsection{Indexing Historical Data}
challenges: growing history


\subsubsection{Indexing Current Data}
challenges: most recent copy

\subsubsection{Indexing Current and Future Positions}
The main focus of the variants in this section is to keep track of the current
and future locations of objects. This is suitable for worksets in which 
objects have predictable behavior such as projectile motion and traveling in
straight lines.

Challenges:

\subsubsection{Database Operations}
Querying moving data?
Clustering


\subsection{Applications}
R-tree variants are used in a vast variety of different applications including
but not exclusive to spatial datamining, geographic information systems, 
science and medicine, and cloud computing. We list some examples of R-trees
and variants used in these fields below.

\begin{description}
	\item[Spatial datamining]
	
	\item[Geographic information systems]
	\item[Science and medicine]
	\item[Cloud computing]
\end{description}

R-trees are also available in some form in many popular open source and 
commercial databases including MySQL Server, SQLite, Oracle$\textsuperscript{\textregistered}$ Spatial, IBM$\textsuperscript{\textregistered}$ DB2, and 
PostgreSQL. 


\section{Conclusion}
\label{sec:conc}
\rbase-trees were introduced as a purely heuristic-based data structure for satisfying range queries.
Their success for that and many other spatial queries has prompted intense study of why it works.
This has in turn improved our understanding of spatial data.
More immediately, a veritable zoo of R-tree variants and improvements exist, both testament to the value of the underlying idea of R-trees as well as the great need for spatial queries.
By presenting the core R-tree, the attempts at improving and understanding them, and the many variants since developed, we hope to provide a starting point to exploring this exciting topic.



% Bibliography section
\bibliographystyle{plain}
\bibliography{rtreepapers}

\end{document}
