The discussion thus far has focused on spatial data in the domain of a single
process. We now extend the scope of our survey of R-trees into the topics of
parallel systems, spatiotemporal databases, and emerging applications.

% Went for a weird title..... can be changed :P
\subsection{R-trees in the Parallel Universe}
Before we begin our overview of R-trees in a parallel environment, we first 
explain some basic definitions. There are three levels of resource sharing in
a parallel system: shared everything, shared disk, and shared nothing
\cite{theBook}. Shared everything means that all processors share all 
resources including disk and memory. Shared disk means that each processor 
has its own private memory but share disks. Unsurprisingly, shared nothing 
architectures have private memory and disk for each processor, which
communicate using some type of network. Thus the basic challenge for R-tree 
implementation in such systems boils down to exploiting parallelism to 
increase both CPU and I/O performance. In the following sections we discuss
methods that are used in multidisk, single-cpu systems, R-tree variants used 
in multiprocessor systems, and operations in parallel databases.

\subsubsection{Multidisk Systems}

\subsubsection{Multiprocessor Systems}

\subsubsection{Operations}

\begin{description}
	\item[yoyoyo] hello
\end{description}

\subsection{R-trees in Spatiotemporal Databases}
There is extensive research for using R-tree variants in spatiotemporal
databases. Essentially spatiotemporal databases add time as an extra dimension
in addition to spatial data. This added dimensionality creates new challenges
[add more to this]. 

The R-tree variants used in spatiotemporal databases can be categorized based
on the time used for indexing data. Structures may index historical 
spatiotemporal data, current data, or future data. These are described more
in detail below.

\subsubsection{Indexing Historical Data}

\subsubsection{Indexing Current Data}

\subsubsection{Indexing Current and Future Positions}
The main focus of the variants in this section is to keep track of the current
and future locations of objects. This is suitable for worksets in which 
objects have predictable behavior such as projectile motion and traveling in
straight lines.

\subsubsection{Operations}
Querying moving data?
Clustering

\subsection{Applications}
\begin{description}
	\item[Spatial datamining]
	\item[Geospatial data / geographic information systems]
	\item[Science and medicine]
	\item[Cloud computing]
	\item[Commercial Systems] Oracle, IBM, PostGreSQL, SQLite
\end{description}

