Although they are by no means the only method for accessing spatial data, the \rbase-tree and 
its many variants are ubiquitous in spatial databases. They are 
used in scientific and medical research, geographic information systems, spatial data mining, 
computer-aided design, commercial systems, and more. Other spatial access methods include 
quadtrees, k-d-B trees, and multidimensional hashing\cite{samet95, gaedegunther98}.
However, we focus our survey on \rbase-trees and relevant topics such as the details of 
the \rbase-tree's basic algorithms and some of its implementation issues. We also discuss 
how \rbase-trees are used in spatial queries, optimization of \rbase-trees, and the 
\rbase-tree cost model. Given its prevalence in spatial databases, we conclude our paper with 
an overview of \rbase-trees in parallel systems, spatio-temporal databases, and other applications. 

Other work that discuss \rbase-trees include surveys on multidimensional access methods
by Gaede and G\"{u}nther \cite{gaedegunther98} and Samet\cite{samet95}. These papers
discuss \rbase-trees and their variants only on a perfunctory level since the scope of their 
work encompasses all types of spatial access methods. A fairly exhaustive book on 
\rbase-trees and its variants has been previously published by Manolopoulos et al.
in \cite{thebook} that covers most of the work on \rbase-trees before 2006. Our survey 
extends these works with an in-depth discussion on more recent \rbase-tree variants as 
well as newer optimizations and solutions to \rbase-tree implementation.

The rest of the paper is organized as follows. In Section~\ref{sec:overview} we discuss the
original \rbase-tree and it main variant, the \rstar-tree. In Section~\ref{sec:impchal}
we examine basic queries using R-trees and associated optimizations. In 
Section~\ref{sec:dbchal} we talk about topics related to \rbase-tree performance, and 
finally in Section~\ref{sec:apps} we extend our discussion to the realm of parallel and
distributed systems, as well as various real-world applications of \rbase-trees.


